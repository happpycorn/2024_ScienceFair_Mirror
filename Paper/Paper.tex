\documentclass[12pt]{article} % 設定文件類型為文章,字型大小為 12pt

\usepackage[utf8]{inputenc}    % 設定字符編碼為 UTF-8
\usepackage{amsmath}           % 載入數學公式支持
\usepackage{graphicx}          % 載入插入圖片的包
\usepackage{geometry}          % 設定頁面邊界

% 設定頁面邊界
\geometry{top=1in, bottom=1in, left=1in, right=1in}

\title{文章標題}   % 設定文章標題
\author{你的名字}   % 設定作者名稱
\date{\today}        % 設定日期為今天

\begin{document}

\maketitle   % 生成標題

\begin{abstract}
這是一篇範例文章的摘要。這裡可以簡要介紹文章內容。
\end{abstract}

\section{引言}
這是文章的引言部分。你可以在這裡介紹你的研究問題、背景和目的。

\section{方法}
在這一部分,你可以描述你的研究方法和步驟。這裡可以包括數學公式,例如:

\begin{equation}
E = mc^2
\end{equation}

這是著名的質能等價公式。

\section{結果}
在這部分,你可以呈現你的研究結果,並進行分析。如果有圖表,也可以插入,例如:

\begin{figure}[h!]
\centering
\includegraphics[width=0.5\textwidth]{example.jpg}
\caption{範例圖片}
\end{figure}

\section{結論}
總結你的研究結果,並提出未來可能的研究方向或建議。

\end{document}
