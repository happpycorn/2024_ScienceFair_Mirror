\documentclass[12pt]{article}
\usepackage[utf8]{inputenc}     % 設定字符編碼為 UTF-8
\usepackage{fontspec}           % 支持自定義字體
\usepackage{setspace}           % 行距設定
\usepackage{titlesec}           % 標題樣式設定
\usepackage{amsmath}            % 支持數學公式
\usepackage{xeCJK}              % 支持 CJK 字體

\usepackage{graphicx}           % 載入插入圖片的包

\usepackage[a4paper,top=2cm,bottom=2cm,left=2cm,right=2cm]{geometry}

\setmainfont{Times New Roman}   % 設定英文主字體(可選)
\setCJKmainfont{DFKai-SB}[
    AutoFakeBold=true,          % 啟用粗體模擬
    AutoFakeSlant=true          % 啟用斜體模擬
]

\onehalfspacing{}               % 設置 1.5 倍行距

% 設定 \section 的字型大小和居中,不顯示節號
\titleformat{\section}[block]
  {\normalfont\fontsize{16}{24}\selectfont\centering}  % 設定字型大小為 16pt,行距為 24pt,並居中顯示
  {}{0em}{}  % 不顯示節號,並設定標題與上方的間距

% 設定 \subsection 與正文字體樣式相同,不顯示節號
\titleformat{\subsection}[block]
  {\normalfont\mdseries\upshape}   % 設定與正文相同的字型樣式
  {}{0em}{}  % 不顯示子節號,並設定標題與上方的間距

% 設定 \subsubsection 與正文字體樣式相同,並加上縮排,不顯示節號
\titleformat{\subsubsection}[block]
  {\normalfont\mdseries\upshape\leftskip=2em}   % 設定與正文相同的字型樣式,並設定左邊縮排 2em
  {}{0em}{}  % 不顯示子子節號,並設定標題與上方的間距

% 設定 \subsubsection 與正文字體樣式相同,並加上縮排,不顯示節號
\titleformat{\subsubsubsection}[block]
  {\normalfont\mdseries\upshape\leftskip=4em}   % 設定與正文相同的字型樣式,並設定左邊縮排 2em
  {}{0em}{}  % 不顯示子子節號,並設定標題與上方的間距

\setlength{\parindent}{2em}  % 設定段落的首行縮排為 2em

\begin{document}

\section{摘要}

\newpage
\section{壹、前言}

\subsection{一、研究動機}
\

在高一時,我實作了一個專案叫做「數位鏡面」。這件作品是模仿一件展出於桃園機場捷運的藝術品,他先透過鏡頭捕捉現實世界的畫面,再透過電腦計算的方式將畫面呈現到畫面上,而我模仿的是其中的一幅,一面透過類似線條的感覺呈現畫面的鏡子。

在實現的過程中我發現,有許多的參數可以影響這面鏡子,表層的有線條的長度與寬度等,而底層的則有每次刷新要畫多少條線等,甚至在不同的使用環境,數位鏡面也會有不同的表現。這引起了我的好奇心,使我想深入探討不同的參數對於鏡子會產生的影響。

\subsection{二、研究目的}

\subsubsection{(一)了解不同參數設定對於數位鏡面的影響}
\subsubsection{(二)了解不同影像環境對於數位鏡面的影響}
\subsubsection{(三)探討不同環境與需求下數位鏡面的最優設定}

\subsection{三、文獻探討}

\subsubsection{(一)數位鏡面}
\

數位鏡面的來源

在這個鏡子中 印出鏡面的步驟 讀取影像、繪製數條線、刷新畫面

繪製線的過程中包含 虛擬一條線、取值、繪製

有三項參數 width, lenth, resolution

\subsubsection{(二)時間複雜度}
\

什麼是時間複雜度?

\subsubsection{(三)互動藝術}
\

數位鏡面等數位藝術究竟在畫什麼?

數位鏡面想要達到怎樣的效果?

\subsubsection{(四)Sum Of Absolute Difference}
\

照片相似度比較

\newpage
\section{貳、研究設備及器材}
\

硬體

電腦乙台

軟體

Python、Opencv、Numpy、Plt、Anaconda、Ipynb

\newpage
\section{參、研究過程與方法}

\subsection{一、研究架構圖}

\subsection{二、研究一:數位鏡面的時間複雜度}

\subsubsection{(一)數位鏡面的時間複雜度計算}

\subsubsection{(二)計算結果驗算}

\subsection{三、研究二:數位鏡面輸出結果量化}

\subsubsection{(一)「模糊」的量化}

\subsubsection{(二)「變化」的量化}

\subsection{四、實驗一:各項參數對於數位鏡面的影響}

\subsubsection{(一)線條寬度(width)對於數位鏡面的影響}

\subsubsection{(二)線條長度(lenth)對於數位鏡面的影響}

\subsubsection{(三)解析度(resolution)對於數位鏡面的影響}

\subsection{五、實驗二:不同場景對於數位鏡面的影響}

\subsubsection{(一)背景顏色對於數位鏡面的影響}
\

顏色差異與顏色數量

\subsubsection{(二)線條長度(lenth)對於數位鏡面的影響}

\subsubsection{(三)解析度(resolution)對於數位鏡面的影響}

\newpage
\section{肆、研究結果}

\subsection{一、研究一:數位鏡面的時間複雜度}

\subsubsection{(一)數位鏡面的時間複雜度計算}

\subsubsection{(二)計算結果驗算}

\subsection{二、研究二:數位鏡面輸出結果量化}

\subsubsection{(一)「模糊」的量化}

\subsubsection{(二)「變化」的量化}

\subsection{三、實驗一:各項參數對於數位鏡面的影響}

\subsubsection{(一)線條寬度(width)對於數位鏡面的影響}

\subsubsection{(二)線條長度(lenth)對於數位鏡面的影響}

\subsubsection{(三)解析度(resolution)對於數位鏡面的影響}

\subsection{四、實驗二:不同場景對於數位鏡面的影響}

\subsubsection{(一)背景顏色對於數位鏡面的影響}

\subsubsection{(二)線條長度(lenth)對於數位鏡面的影響}

\subsubsection{(三)解析度(resolution)對於數位鏡面的影響}

\newpage
\section{伍、討論}

\newpage
\section{陸、結論}

\newpage
\section{柒、參考文獻資料}

github 我的數位鏡面

\end{document}